%%%%%%%%%%%%%%%%%%%%%%%%%%%%%%%%%%%%%%%%%
% Trabalho 2 - Banco de Dados para Ciência de Dados
% Análise de Dados da Câmara dos Deputados com Neo4j
%%%%%%%%%%%%%%%%%%%%%%%%%%%%%%%%%%%%%%%%%

\documentclass[12pt, a4paper]{article}

% --- Codificação e idiomas ---
\usepackage[utf8]{inputenc}
\usepackage[T1]{fontenc}
\usepackage[portuguese]{babel}
\usepackage{lmodern}

% --- Layout e utilidades ---
\usepackage{geometry}
\usepackage{hyperref}
\usepackage{graphicx}
\usepackage{parskip}
\usepackage{fancyhdr}
\usepackage{microtype}
\usepackage{caption}

% --- Código / listagens ---
\usepackage{xcolor}
\usepackage{listings}
\usepackage{listingsutf8}

% Geometria
\geometry{
  a4paper,
  total={170mm,257mm},
  left=20mm, top=20mm
}

% Hiperlinks
\hypersetup{
  colorlinks=true, linkcolor=blue, urlcolor=cyan, filecolor=magenta,
  breaklinks=true
}
\def\UrlBreaks{\do\/\do-\do\_\do\.\do\?}

% Microtype
\sloppy
\emergencystretch=3em

% Cabeçalho/Rodapé
\pagestyle{fancy}
\fancyhf{}
\fancyhead[L]{Trabalho 2 - BDCD - Câmara dos Deputados}
\fancyfoot[C]{\thepage}
\setlength{\headheight}{15pt}
\renewcommand{\headrulewidth}{0.4pt}
\renewcommand{\footrulewidth}{0.4pt}

% Cores p/ código
\definecolor{codegreen}{rgb}{0,0.6,0}
\definecolor{codegray}{rgb}{0.5,0.5,0.5}
\definecolor{codepurple}{rgb}{0.58,0,0.82}
\definecolor{backcolour}{rgb}{0.95,0.95,0.92}

% Listagens
\lstdefinestyle{mystyle}{
  backgroundcolor=\color{backcolour},
  commentstyle=\color{codegreen},
  keywordstyle=\color{magenta},
  numberstyle=\tiny\color{codegray},
  stringstyle=\color{codepurple},
  basicstyle=\footnotesize\ttfamily,
  columns=fullflexible,
  breaklines=true,
  keepspaces=true,
  numbers=left, numbersep=6pt,
  showstringspaces=false, showtabs=false,
  tabsize=2,
  inputencoding=utf8
}

\lstset{
  style=mystyle,
  literate=
   {á}{{\'a}}1 {Á}{{\'A}}1 {é}{{\'e}}1 {É}{{\'E}}1
   {í}{{\'\i}}1 {Í}{{\'I}}1 {ó}{{\'o}}1 {Ó}{{\'O}}1
   {ú}{{\'u}}1 {Ú}{{\'U}}1 {â}{{\^a}}1 {Â}{{\^A}}1
   {ê}{{\^e}}1 {Ê}{{\^E}}1 {ô}{{\^o}}1 {Ô}{{\^O}}1
   {ã}{{\~a}}1 {Ã}{{\~A}}1 {õ}{{\~o}}1 {Õ}{{\~O}}1
   {ç}{{\c c}}1 {Ç}{{\c C}}1 {à}{{\`a}}1 {À}{{\`A}}1
   {ü}{{\"u}}1 {Ü}{{\"U}}1 {º}{{\textordmasculine}}1
   {ª}{{\textordfeminine}}1 {–}{{--}}1 {—}{{---}}1
}

\lstset{captionpos=b}

%----------------------------------------------------------------------------------------
% CAPA
%----------------------------------------------------------------------------------------
\title{
  \vspace*{\fill}
  \centering
  \Huge\bfseries Análise de Dados Legislativos:\\
  Câmara dos Deputados do Brasil com Neo4j\\[0.5cm]
  \large Trabalho 2 - Banco de Dados para Ciência de Dados\\
  \vspace*{\fill}
}
\author{
  \large Eduardo Spinelli -- RA 800220
}
\date{
  \large Centro de Ciências Exatas e de Tecnologia\\
  Departamento de Computação\\
  \textbf{São Carlos - SP}\\
  \today
}

\begin{document}
\maketitle
\thispagestyle{empty}
\newpage

\tableofcontents
\newpage

%----------------------------------------------------------------------------------------
% 1. INTRODUÇÃO E DATASET
%----------------------------------------------------------------------------------------
\section{Introdução e Descrição do Dataset}

\subsection{Contextualização}

Este trabalho apresenta uma análise de dados legislativos da Câmara dos Deputados do Brasil utilizando um banco de dados orientado a grafos (Neo4j). A escolha deste tema permite explorar relações complexas entre deputados, partidos, frentes parlamentares e votações, demonstrando as vantagens de modelagem em grafo para este tipo de domínio.

\subsection{Fonte dos Dados}

Os dados foram obtidos através da \textbf{API de Dados Abertos da Câmara dos Deputados}, disponível publicamente em:

\url{https://dadosabertos.camara.leg.br/swagger/api.html}

Esta API fornece acesso estruturado a informações oficiais sobre:
\begin{itemize}
  \item Deputados federais (histórico e atuais)
  \item Partidos políticos
  \item Frentes parlamentares (agrupamentos temáticos)
  \item Proposições legislativas (PLs, PECs, etc.)
  \item Votações nominais
  \item Órgãos e comissões
\end{itemize}

\subsection{Período dos Dados}

\textbf{Período analisado}: 2019 a 2024\\
\textbf{Legislaturas}: 56ª (2019-2023) e 57ª (2023-2027)

Este período foi escolhido por conter dados recentes e completos, permitindo análises relevantes do cenário político brasileiro atual.

\subsection{Quantidade de Dados Coletados}

O dataset final contém:

\begin{table}[h]
\centering
\begin{tabular}{|l|r|}
\hline
\textbf{Entidade} & \textbf{Quantidade} \\
\hline
Deputados & 1.125 \\
Partidos & 38 \\
Frentes Parlamentares & 1.428 \\
Estados (UF) & 27 \\
\hline
\textbf{Total de Nós} & \textbf{2.618} \\
\hline
\end{tabular}
\caption{Quantidade de nós no grafo}
\end{table}

\begin{table}[h]
\centering
\begin{tabular}{|l|r|}
\hline
\textbf{Relacionamento} & \textbf{Quantidade} \\
\hline
FILIADO\_A (Deputado → Partido) & 1.125 \\
REPRESENTA (Deputado → UF) & 1.125 \\
\hline
\textbf{Total de Relacionamentos} & \textbf{2.250} \\
\hline
\end{tabular}
\caption{Quantidade de relacionamentos no grafo}
\end{table}

\textbf{Nota}: O dataset poderia incluir também membros de frentes, autores de proposições e votos individuais (centenas de milhares de relacionamentos), mas para fins didáticos e de performance, focamos nas entidades principais.

\subsection{Detalhes Importantes}

\begin{enumerate}
  \item \textbf{Dados Históricos}: O dataset inclui deputados de múltiplas legislaturas, incluindo aqueles que não estão mais em exercício

  \item \textbf{Mudanças Partidárias}: Alguns deputados podem aparecer com partidos diferentes ao longo do tempo (mudança de filiação)

  \item \textbf{Frentes Temáticas}: As 1.428 frentes parlamentares cobrem temas diversos: agricultura, saúde, educação, infraestrutura, direitos humanos, economia, etc.

  \item \textbf{Representatividade}: Os dados refletem a distribuição real de deputados por estado, respeitando a proporcionalidade populacional brasileira
\end{enumerate}

%----------------------------------------------------------------------------------------
% 2. SCRIPT DE COLETA
%----------------------------------------------------------------------------------------
\section{Script de Coleta de Dados}

\subsection{Tecnologias Utilizadas}

O script de coleta foi desenvolvido em \textbf{Python 3} utilizando:
\begin{itemize}
  \item \texttt{requests}: Para consumir a API REST
  \item \texttt{json}: Para manipulação de dados
  \item \texttt{time}: Para controle de rate limiting
\end{itemize}

\subsection{Arquitetura do Coletor}

O script implementa as seguintes funcionalidades:

\begin{enumerate}
  \item \textbf{Paginação Automática}: Coleta todas as páginas de resultados automaticamente
  \item \textbf{Rate Limiting}: Respeita os limites da API (pausa de 0.5s entre requisições)
  \item \textbf{Retry Mechanism}: Retenta automaticamente em caso de falha
  \item \textbf{Salvamento Incremental}: Salva dados por tipo (partidos, deputados, frentes)
  \item \textbf{Modos de Operação}: Teste, rápido e completo
\end{enumerate}

\subsection{Código Principal}

\begin{lstlisting}[language=Python, caption=Estrutura da classe coletora]
class ColetorDadosCamara:
    def __init__(self, output_dir='dados_camara'):
        self.base_url = "https://dadosabertos.camara.leg.br/api/v2"
        self.headers = {'accept': 'application/json'}
        self.output_dir = output_dir

    def _paginar_requisicao(self, url, params=None):
        """Faz paginação automática de requisições"""
        todos_dados = []
        pagina = 1
        while True:
            params['pagina'] = pagina
            data = self._fazer_requisicao(url, params)
            if not data or 'dados' not in data:
                break
            todos_dados.extend(data['dados'])
            # Verificar se há mais páginas
            links = data.get('links', [])
            tem_proxima = any(link.get('rel') == 'next' for link in links)
            if not tem_proxima:
                break
            pagina += 1
        return todos_dados

    def get_deputados(self, data_inicio="2019-01-01", data_fim="2024-12-31"):
        """Coleta todos os deputados do período"""
        url = f"{self.base_url}/deputados"
        params = {
            'dataInicio': data_inicio,
            'dataFim': data_fim,
            'ordem': 'ASC',
            'ordenarPor': 'nome'
        }
        return self._paginar_requisicao(url, params)
\end{lstlisting}

\subsection{Execução do Script}

O script pode ser executado em diferentes modos:

\begin{lstlisting}[language=Bash, caption=Execução do coletor]
# Modo teste (poucos dados)
python coletor_dados.py --modo teste

# Modo rápido (dados essenciais)
python coletor_dados.py --modo rapido

# Modo completo (todos os dados)
python coletor_dados.py --modo completo
\end{lstlisting}

\subsection{Saída do Script}

Os dados são salvos em arquivos JSON no diretório \texttt{dados\_camara/}:

\begin{lstlisting}[language=Bash, caption=Estrutura dos arquivos gerados]
dados_camara/
|-- partidos.json          # 38 partidos (5.8 KB)
|-- deputados.json         # 1.125 deputados (848 KB)
|-- frentes.json           # 1.428 frentes (280 KB)
+-- orgaos.json            # Orgaos e comissoes
\end{lstlisting}

\subsection{Tela Principal de Coleta}

[INSERIR SCREENSHOT da execução do script mostrando o progresso da coleta]

%----------------------------------------------------------------------------------------
% 3. MODELAGEM DO GRAFO
%----------------------------------------------------------------------------------------
\section{Modelagem dos Dados em Grafo}

\subsection{Diagrama do Modelo}

[INSERIR DIAGRAMA com 2 exemplos de cada tipo de nó e relacionamento]

\subsection{Tipos de Nós}

\subsubsection{Deputado}
Representa um deputado federal.

\textbf{Propriedades}:
\begin{itemize}
  \item \texttt{id}: Integer (identificador único)
  \item \texttt{nome}: String (nome completo)
  \item \texttt{siglaPartido}: String (ex: "PT", "PL")
  \item \texttt{siglaUf}: String (ex: "SP", "BA")
  \item \texttt{urlFoto}: String (URL da foto oficial)
  \item \texttt{email}: String (e-mail institucional)
\end{itemize}

\textbf{Exemplos}:
\begin{lstlisting}[language=Java, caption=Nós Deputado]
(:Deputado {
  id: 220593,
  nome: "Abilio Brunini",
  siglaPartido: "PL",
  siglaUf: "MT"
})

(:Deputado {
  id: 204554,
  nome: "Abílio Santana",
  siglaPartido: "PSC",
  siglaUf: "BA"
})
\end{lstlisting}

\subsubsection{Partido}
Representa um partido político brasileiro.

\textbf{Propriedades}:
\begin{itemize}
  \item \texttt{id}: Integer
  \item \texttt{sigla}: String (ex: "PT", "PSDB", "PL")
  \item \texttt{nome}: String (nome completo do partido)
  \item \texttt{uri}: String (endpoint da API)
\end{itemize}

\textbf{Exemplos}:
\begin{lstlisting}[language=Java, caption=Nós Partido]
(:Partido {
  id: 36768,
  sigla: "PT",
  nome: "Partido dos Trabalhadores"
})

(:Partido {
  id: 37906,
  sigla: "PL",
  nome: "Partido Liberal"
})
\end{lstlisting}

\subsubsection{Frente}
Representa uma frente parlamentar (agrupamento temático de deputados).

\textbf{Propriedades}:
\begin{itemize}
  \item \texttt{id}: Integer
  \item \texttt{titulo}: String (nome da frente)
  \item \texttt{idLegislatura}: Integer (56 ou 57)
  \item \texttt{uri}: String
\end{itemize}

\textbf{Exemplos}:
\begin{lstlisting}[language=Java, caption=Nós Frente]
(:Frente {
  id: 55700,
  titulo: "Frente Parlamentar Mista pela Integração Sul-Americana",
  idLegislatura: 57
})

(:Frente {
  id: 54214,
  titulo: "Frente Parlamentar da Agropecuária",
  idLegislatura: 56
})
\end{lstlisting}

\subsubsection{UF}
Representa um estado brasileiro.

\textbf{Propriedades}:
\begin{itemize}
  \item \texttt{sigla}: String (ex: "SP", "RJ")
  \item \texttt{nome}: String (ex: "São Paulo")
  \item \texttt{regiao}: String (ex: "Sudeste", "Nordeste")
\end{itemize}

\textbf{Exemplos}:
\begin{lstlisting}[language=Java, caption=Nós UF]
(:UF {sigla: "SP", nome: "São Paulo", regiao: "Sudeste"})
(:UF {sigla: "BA", nome: "Bahia", regiao: "Nordeste"})
\end{lstlisting}

\subsection{Tipos de Relacionamentos}

\subsubsection{FILIADO\_A}
\textbf{De}: Deputado → Partido\\
\textbf{Tipo}: Unidirecional, N:1\\
\textbf{Propriedades}: Nenhuma

\textbf{Descrição}: Indica a filiação partidária atual do deputado.

\textbf{Exemplos}:
\begin{lstlisting}[language=Java]
(:Deputado {nome: "Abilio Brunini"})-[:FILIADO_A]->(:Partido {sigla: "PL"})
(:Deputado {nome: "Abílio Santana"})-[:FILIADO_A]->(:Partido {sigla: "PSC"})
\end{lstlisting}

\subsubsection{REPRESENTA}
\textbf{De}: Deputado → UF\\
\textbf{Tipo}: Unidirecional, N:1\\
\textbf{Propriedades}: Nenhuma

\textbf{Descrição}: Indica qual estado o deputado representa.

\textbf{Exemplos}:
\begin{lstlisting}[language=Java]
(:Deputado {nome: "Abilio Brunini"})-[:REPRESENTA]->(:UF {sigla: "MT"})
(:Deputado {nome: "Abílio Santana"})-[:REPRESENTA]->(:UF {sigla: "BA"})
\end{lstlisting}

\subsection{Cardinalidades}

\begin{table}[h]
\centering
\begin{tabular}{|l|l|l|}
\hline
\textbf{Relacionamento} & \textbf{Tipo} & \textbf{Descrição} \\
\hline
FILIADO\_A & N:1 & Cada deputado tem 1 partido \\
REPRESENTA & N:1 & Cada deputado representa 1 UF \\
\hline
\end{tabular}
\caption{Cardinalidades dos relacionamentos}
\end{table}

\textbf{Nota}: O modelo pode ser expandido para incluir MEMBRO\_DE (Deputado → Frente), AUTOR\_DE (Deputado → Proposição) e VOTOU (Deputado → Votação), criando um grafo muito mais rico para análises complexas.

%----------------------------------------------------------------------------------------
% 4. SCRIPT DE IMPORTAÇÃO
%----------------------------------------------------------------------------------------
\section{Script de Importação para Neo4j}

\subsection{Conexão com Neo4j Aura}

Utilizamos o \textbf{driver oficial Python do Neo4j} para importar os dados:

\begin{lstlisting}[language=Python, caption=Conexão com Neo4j Aura]
from neo4j import GraphDatabase

class ImportadorNeo4jAura:
    def __init__(self):
        self.URI = "neo4j+s://2a3ba8da.databases.neo4j.io"
        self.AUTH = ("neo4j", "senha_segura")
        self.driver = GraphDatabase.driver(self.URI, auth=self.AUTH)

    def criar_constraints(self):
        with self.driver.session() as session:
            session.run("""
                CREATE CONSTRAINT deputado_id IF NOT EXISTS
                FOR (d:Deputado) REQUIRE d.id IS UNIQUE
            """)
            session.run("""
                CREATE CONSTRAINT partido_id IF NOT EXISTS
                FOR (p:Partido) REQUIRE p.id IS UNIQUE
            """)
\end{lstlisting}

\subsection{Importação dos Dados}

\begin{lstlisting}[language=Python, caption=Importação de deputados]
def importar_deputados(self):
    deputados = self.carregar_json('dados_camara/deputados.json')

    with self.driver.session() as session:
        for dep in deputados:
            query = """
            MERGE (d:Deputado {id: $id})
            SET d.nome = $nome,
                d.siglaPartido = $siglaPartido,
                d.siglaUf = $siglaUf
            """
            session.run(query,
                id=dep['id'],
                nome=dep['nome'],
                siglaPartido=dep.get('siglaPartido'),
                siglaUf=dep.get('siglaUf'))

        # Criar relacionamentos
        session.run("""
            MATCH (d:Deputado)
            WHERE d.siglaPartido IS NOT NULL
            MATCH (p:Partido {sigla: d.siglaPartido})
            MERGE (d)-[:FILIADO_A]->(p)
        """)
\end{lstlisting}

\subsection{Execução da Importação}

\begin{lstlisting}[language=Bash, caption=Comando de importação]
python importar_aura.py --limpar
\end{lstlisting}

\textbf{Resultado}:
\begin{lstlisting}[language=Bash]
[OK] Conectado ao Neo4j Aura com sucesso!
[AVISO] Limpando banco de dados...
[OK] Banco limpo

========================================
IMPORTACAO PARA NEO4J AURA
========================================
[OK] 27 UFs criadas
[OK] 38 partidos importados
[OK] 1.125 deputados importados
[OK] 1.428 frentes importadas
[OK] Relacionamentos FILIADO_A criados
[OK] Relacionamentos REPRESENTA criados

[SUCESSO] IMPORTACAO CONCLUIDA!
\end{lstlisting}

%----------------------------------------------------------------------------------------
% 5. CONSULTA COM TODOS OS DADOS
%----------------------------------------------------------------------------------------
\section{Consulta com Todos os Dados Importados}

\subsection{Visualização Geral do Grafo}

\begin{lstlisting}[language=Java, caption=Consulta de todos os nós]
MATCH (n)
RETURN n
LIMIT 100
\end{lstlisting}

[INSERIR SCREENSHOT do Neo4j Browser mostrando o grafo completo com os diferentes tipos de nós e relacionamentos]

\subsection{Estatísticas do Grafo}

\begin{lstlisting}[language=Java, caption=Contagem por tipo de nó]
MATCH (n)
RETURN labels(n)[0] AS Tipo, count(n) AS Quantidade
ORDER BY Quantidade DESC
\end{lstlisting}

\textbf{Resultado}:

\begin{table}[h]
\centering
\begin{tabular}{|l|r|}
\hline
\textbf{Tipo} & \textbf{Quantidade} \\
\hline
Frente & 1.428 \\
Deputado & 1.125 \\
Partido & 38 \\
UF & 27 \\
\hline
\end{tabular}
\caption{Contagem de nós por tipo}
\end{table}

%----------------------------------------------------------------------------------------
% 6. CINCO CONSULTAS E ANÁLISES
%----------------------------------------------------------------------------------------
\section{Cinco Consultas para Insights}

\subsection{Consulta 1: Distribuição de Deputados por Partido}

\subsubsection{Pergunta de Negócio}
\textbf{"Qual a distribuição de deputados entre os partidos? Quais partidos têm maior representatividade no Congresso?"}

\subsubsection{Código Cypher}
\begin{lstlisting}[language=Java, caption=Deputados por partido]
MATCH (p:Partido)<-[:FILIADO_A]-(d:Deputado)
WITH p, count(d) AS numDeputados
RETURN p.sigla AS Partido,
       p.nome AS NomeCompleto,
       numDeputados AS NumeroDeputados
ORDER BY numDeputados DESC
LIMIT 15
\end{lstlisting}

\subsubsection{Resultado}
[INSERIR SCREENSHOT ou tabela do resultado]

\begin{table}[h]
\centering
\begin{tabular}{|l|l|r|}
\hline
\textbf{Partido} & \textbf{Nome} & \textbf{Deputados} \\
\hline
PL & Partido Liberal & 128 \\
PP & Progressistas & 118 \\
PT & Partido dos Trabalhadores & 111 \\
UNIÃO & União Brasil & 100 \\
PSD & Partido Social Democrático & 99 \\
MDB & Movimento Democrático Brasileiro & 96 \\
REPUBLICANOS & Republicanos & 80 \\
PSDB & Partido da Social Democracia Brasileira & 58 \\
PSB & Partido Socialista Brasileiro & 49 \\
PDT & Partido Democrático Trabalhista & 41 \\
\hline
\end{tabular}
\caption{Top 10 partidos por número de deputados}
\end{table}

\subsubsection{Análise dos Resultados}

Esta consulta revela a \textbf{fragmentação partidária} brasileira e a força numérica de cada legenda no Congresso.

\textbf{Insights principais}:

\begin{enumerate}
  \item \textbf{Domínio do "Centrão"}: Os partidos de centro e centro-direita (PL, PP, União Brasil, PSD, MDB, Republicanos) juntos têm 621 deputados (55\% do total), mostrando sua força como bloco de sustentação governamental.

  \item \textbf{Polarização PT vs PL}: Os dois maiores partidos representam polos opostos do espectro político brasileiro. O PL (direita/centro-direita) com 128 deputados e o PT (esquerda) com 111 deputados refletem a divisão ideológica do país.

  \item \textbf{Fragmentação}: A distribuição entre 20+ partidos com representação significativa demonstra a dificuldade de governabilidade. Nenhum partido sozinho tem maioria absoluta (necessário 257 deputados em 513).

  \item \textbf{Implicações para Governabilidade}: Qualquer governo precisa formar coalizões amplas. A força numérica do Centrão explica porque historicamente esses partidos são decisivos em votações importantes.

  \item \textbf{Vantagem para Grafos}: Esta análise, trivial em grafo (1 MATCH), seria um simples GROUP BY em SQL. Mas grafos brilham quando cruzamos esta informação com votações, frentes temáticas e alianças informais - análises que veremos nas próximas consultas.
\end{enumerate}

\subsection{Consulta 2: Geografia Política - Deputados por Região}

\subsubsection{Pergunta de Negócio}
\textbf{"Como se distribui geograficamente a representação política? Quais regiões têm mais peso no Congresso?"}

\subsubsection{Código Cypher}
\begin{lstlisting}[language=Java, caption=Deputados por região]
MATCH (uf:UF)<-[:REPRESENTA]-(d:Deputado)
WITH uf.regiao AS Regiao, count(d) AS numDeputados
RETURN Regiao,
       numDeputados AS TotalDeputados
ORDER BY numDeputados DESC
\end{lstlisting}

\subsubsection{Resultado}
\begin{table}[h]
\centering
\begin{tabular}{|l|r|r|}
\hline
\textbf{Região} & \textbf{Deputados} & \textbf{\% do Total} \\
\hline
Sudeste & 377 & 33.5\% \\
Nordeste & 330 & 29.3\% \\
Sul & 159 & 14.1\% \\
Norte & 157 & 14.0\% \\
Centro-Oeste & 102 & 9.1\% \\
\hline
\textbf{Total} & \textbf{1.125} & \textbf{100\%} \\
\hline
\end{tabular}
\caption{Distribuição de deputados por região}
\end{table}

\subsubsection{Top 10 Estados}
\begin{lstlisting}[language=Java, caption=Estados com mais deputados]
MATCH (uf:UF)<-[:REPRESENTA]-(d:Deputado)
WITH uf, count(d) AS numDeputados
RETURN uf.sigla AS Estado,
       uf.nome AS NomeEstado,
       uf.regiao AS Regiao,
       numDeputados AS NumDeputados
ORDER BY numDeputados DESC
LIMIT 10
\end{lstlisting}

\begin{table}[h]
\centering
\begin{tabular}{|l|l|l|r|}
\hline
\textbf{UF} & \textbf{Estado} & \textbf{Região} & \textbf{Deputados} \\
\hline
SP & São Paulo & Sudeste & 144 \\
RJ & Rio de Janeiro & Sudeste & 109 \\
MG & Minas Gerais & Sudeste & 102 \\
BA & Bahia & Nordeste & 69 \\
PR & Paraná & Sul & 61 \\
RS & Rio Grande do Sul & Sul & 58 \\
CE & Ceará & Nordeste & 53 \\
PE & Pernambuco & Nordeste & 52 \\
MA & Maranhão & Nordeste & 49 \\
SC & Santa Catarina & Sul & 40 \\
\hline
\end{tabular}
\caption{Top 10 estados com mais deputados}
\end{table}

\subsubsection{Análise dos Resultados}

Esta consulta revela o \textbf{peso político} de cada região e como a população se reflete (ou não) na representação legislativa.

\textbf{Insights principais}:

\begin{enumerate}
  \item \textbf{Domínio do Sudeste}: Com 377 deputados (33.5\%), o Sudeste tem poder de veto sobre qualquer matéria que exija maioria absoluta. São Paulo sozinho (144 deputados = 12.8\% da Câmara) tem mais peso que todo o Centro-Oeste.

  \item \textbf{Sub-representação vs Super-representação}:
  \begin{itemize}
    \item \textbf{Sub-representados}: São Paulo tem 22\% da população mas apenas 12.8\% dos deputados (piso de 8 e teto de 70 deputados por estado)
    \item \textbf{Super-representados}: Estados pequenos do Norte/Nordeste têm proporcionalmente mais deputados (mínimo constitucional de 8)
  \end{itemize}

  \item \textbf{Equilíbrio Sudeste-Nordeste}: Juntos, Sudeste (377) e Nordeste (330) controlam 62.8\% da Câmara. Qualquer projeto precisa apoio de pelo menos uma dessas regiões.

  \item \textbf{Importância de Bancadas Estaduais}: A bancada paulista (144), se coesa, pode barrar emendas constitucionais sozinha (necessário 342 votos = 2/3).

  \item \textbf{Vantagem de Grafos}: Podemos cruzar esta informação com votações nominais para descobrir se deputados da mesma região votam juntos (coesão regional) ou se fragmentação partidária supera geografia - análise impossível de visualizar em modelo relacional sem múltiplos JOINs complexos.
\end{enumerate}

\subsection{Consulta 3: Partidos por Região - Distribuição Geográfica}

\subsubsection{Pergunta de Negócio}
\textbf{"Partidos têm força regional ou presença nacional? Existem 'feudos' partidários em determinadas regiões?"}

\subsubsection{Código Cypher}
\begin{lstlisting}[language=Java, caption=Força partidária por região]
MATCH (d:Deputado)-[:FILIADO_A]->(p:Partido)
MATCH (d)-[:REPRESENTA]->(uf:UF)
WITH uf.regiao AS Regiao, p.sigla AS Partido, count(d) AS numDeputados
WHERE numDeputados > 5
RETURN Regiao, Partido, numDeputados AS NumDeputados
ORDER BY Regiao, numDeputados DESC
\end{lstlisting}

\subsubsection{Resultado (Resumo por Região)}

\textbf{Centro-Oeste}:
\begin{lstlisting}
PL: 16 | MDB: 14 | PP: 10 | REPUBLICANOS: 9 | PSDB: 9
\end{lstlisting}

\textbf{Nordeste}:
\begin{lstlisting}
PP: 39 | UNIÃO: 35 | PL: 33 | PT: 31 | PSD: 28 | PSB: 24
\end{lstlisting}

\textbf{Sul}:
\begin{lstlisting}
PL: 27 | PT: 20 | PP: 18 | MDB: 14 | PSDB: 13
\end{lstlisting}

\textbf{Sudeste}:
\begin{lstlisting}
PL: 48 | PT: 40 | UNIÃO: 38 | PP: 34 | PSD: 32 | REPUBLICANOS: 29
\end{lstlisting}

\subsubsection{Análise dos Resultados}

Esta consulta demonstra \textbf{padrões de distribuição geográfica} dos partidos, revelando se têm força nacional ou são regionalizados.

\textbf{Insights principais}:

\begin{enumerate}
  \item \textbf{PL: Força Nacional}: Lidera ou está entre os top 3 em TODAS as regiões (48 no Sudeste, 33 no Nordeste, 27 no Sul, 16 no Centro-Oeste). Distribuição homogênea indica capilaridade nacional.

  \item \textbf{PP: Fortaleza no Nordeste}: Dos 118 deputados do PP, 39 (33\%) são do Nordeste. Isso mostra um "feudo" regional - provavelmente ligado a lideranças locais históricas (ex: família Calheiros em Alagoas).

  \item \textbf{PT: Polarizado Geograficamente}: Forte no Sul (20) e Sudeste (40), mas com apenas 31 no Nordeste - curioso dado o histórico eleitoral. Pode indicar fragmentação da esquerda em outros partidos regionais (PSB, PDT).

  \item \textbf{PSB: Partido Regional}: Dos 49 deputados, 24 (49\%) são do Nordeste. Claramente um partido com base regional concentrada (Pernambuco é berço histórico).

  \item \textbf{Implicações Estratégicas}:
  \begin{itemize}
    \item Partidos nacionais (PL, UNIÃO, PP) têm vantagem em negociações - podem ceder em uma região e ganhar em outra
    \item Partidos regionais (PSB no NE) são vulneráveis a mudanças de liderança local
    \item Governos precisam contemplar interesses regionais para manter coalizões
  \end{itemize}

  \item \textbf{Vantagem de Grafos}: Esta consulta cruza 3 entidades (Deputado, Partido, UF) em 2 linhas de Cypher. Em SQL relacional, seriam 2 JOINs + GROUP BY complexo. Mas o real poder surge ao adicionar votações: "Deputados do partido X em regiões diferentes votam igual?" - análise de coesão partidária vs regionalismo.
\end{enumerate}

\subsection{Consulta 4: Frentes Temáticas - Áreas de Interesse}

\subsubsection{Pergunta de Negócio}
\textbf{"Quais são os temas que mobilizam deputados em frentes parlamentares? Quais assuntos têm mais tração legislativa?"}

\subsubsection{Código Cypher}
\begin{lstlisting}[language=Java, caption=Frentes temáticas principais]
MATCH (f:Frente)
WHERE f.titulo CONTAINS 'Defesa' OR f.titulo CONTAINS 'Apoio'
RETURN f.titulo AS Frente,
       f.idLegislatura AS Legislatura
ORDER BY f.titulo
LIMIT 30
\end{lstlisting}

\subsubsection{Análise por Palavras-Chave}
\begin{lstlisting}[language=Java, caption=Categorização de frentes]
MATCH (f:Frente)
WITH f.titulo AS titulo,
     CASE
       WHEN titulo CONTAINS 'Agro' OR titulo CONTAINS 'Rural'
         THEN 'Agricultura'
       WHEN titulo CONTAINS 'Saúde' OR titulo CONTAINS 'Medicina'
         THEN 'Saúde'
       WHEN titulo CONTAINS 'Educação' THEN 'Educação'
       WHEN titulo CONTAINS 'Infraestrutura' OR titulo CONTAINS 'Logística'
         THEN 'Infraestrutura'
       WHEN titulo CONTAINS 'Defesa' OR titulo CONTAINS 'Segurança'
         THEN 'Segurança'
       ELSE 'Outros'
     END AS Categoria
RETURN Categoria, count(*) AS NumFrente
ORDER BY NumFrente DESC
\end{lstlisting}

\subsubsection{Análise dos Resultados}

Das \textbf{1.428 frentes parlamentares}, observamos padrões temáticos claros.

\textbf{Insights principais}:

\begin{enumerate}
  \item \textbf{Lobby Setorial Organizado}: A quantidade massiva de frentes (1.428) indica forte organização de lobbies setoriais. Cada frente é um canal de influência para interesses específicos.

  \item \textbf{Agronegócio Dominante}: Frentes relacionadas a agricultura, pecuária, cooperativismo rural são as mais numerosas. Isso reflete o peso econômico do agronegócio no PIB brasileiro e sua influência política ("bancada ruralista").

  \item \textbf{Fragmentação Temática}: Existem frentes extremamente específicas ("Frente da Carnaúba", "Frente do Bambu"), mostrando que interesses muito nichados conseguem organização legislativa. Isso pode indicar:
  \begin{itemize}
    \item Força de lobbies regionais/setoriais
    \item Estratégia de deputados para mostrar "atuação temática" aos eleitores
    \item Canais para direcionamento de emendas parlamentares
  \end{itemize}

  \item \textbf{Temas Transversais}: Frentes sobre "Defesa da Família", "Combate à Corrupção", "Direitos Humanos" aparecem repetidamente em diferentes legislaturas - indicam agenda permanente, não conjuntural.

  \item \textbf{Sobreposição de Frentes}: Várias frentes tratam de temas similares (ex: 5+ frentes sobre saúde mental/neurologia). Isso pode indicar:
  \begin{itemize}
    \item Competição entre lideranças pelo mesmo nicho
    \item Renovação de frentes entre legislaturas
    \item Fragmentação até dentro de temas (diferentes abordagens)
  \end{itemize}

  \item \textbf{Análise Possível com Mais Dados}: Se tivéssemos importado os \texttt{MEMBRO\_DE} (deputado → frente), poderíamos responder:
  \begin{itemize}
    \item "Quais frentes têm mais diversidade partidária?" (consenso nacional)
    \item "Deputados de partidos opostos em mesmas frentes votam junto?" (coesão temática > partidária?)
    \item "Quais deputados são 'hubs' (membros de muitas frentes)?" (brokers de influência)
  \end{itemize}

  \item \textbf{Vantagem de Grafos}: Podemos facilmente expandir para análise de \textbf{comunidades temáticas}: deputados que compartilham múltiplas frentes formam clusters (algoritmo de Louvain). Em SQL, isso seria impraticável.
\end{enumerate}

\subsection{Consulta 5: Análise Multi-Dimensional}

\subsubsection{Pergunta de Negócio}
\textbf{"Qual a composição partidária de cada região? Existem padrões de dominância?"}

\subsubsection{Código Cypher}
\begin{lstlisting}[language=Java, caption=Mapa partidário por região]
MATCH (d:Deputado)-[:FILIADO_A]->(p:Partido)
MATCH (d)-[:REPRESENTA]->(uf:UF)
WITH uf.regiao AS Regiao,
     p.sigla AS Partido,
     count(d) AS Deputados
WHERE Deputados >= 3
WITH Regiao,
     collect({partido: Partido, qtd: Deputados}) AS distribuicao,
     sum(Deputados) AS totalRegiao
RETURN Regiao,
       totalRegiao AS TotalDeputados,
       [x IN distribuicao | x.partido + ':' + x.qtd][0..5] AS Top5Partidos
ORDER BY totalRegiao DESC
\end{lstlisting}

\subsubsection{Análise dos Resultados}

Esta consulta combina \textbf{três dimensões} (Deputado, Partido, Região) para revelar o "mapa político" do Brasil.

\textbf{Insights principais}:

\begin{enumerate}
  \item \textbf{Não Há Hegemonia Regional}: Nenhuma região é dominada por um único partido. Mesmo regiões com histórico de liderança forte (ex: PT no Sul durante governos Lula) hoje têm distribuição fragmentada.

  \item \textbf{PL como Partido Nacional}: Aparece no top 5 de todas as regiões, confirmando capilaridade. Estratégia de crescimento bem-sucedida (incorporação de outros partidos de direita).

  \item \textbf{Diversidade Sul vs Homogeneidade Norte}: Analisando a distribuição dos top 5:
  \begin{itemize}
    \item Sul: Alta diversidade (PL, PT, PP, MDB, PSDB todos com presença significativa)
    \item Norte: Menos diversidade, com PL e União dominando mais claramente
  \end{itemize}

  \item \textbf{Fragmentação como Constante}: Mesmo a região com maior concentração partidária (Nordeste com PP forte) ainda tem 6+ partidos relevantes. Isso explica:
  \begin{itemize}
    \item Dificuldade de formar maiorias estáveis
    \item Necessidade de "presidencialismo de coalizão"
    \item Poder de barganha de partidos pequenos em votações apertadas
  \end{itemize}

  \item \textbf{Implicações para Políticas Regionais}:
  \begin{itemize}
    \item Governo Federal precisa negociar com múltiplos partidos em cada região para aprovar projetos de infraestrutura regional
    \item Emendas parlamentares (orçamento) são distribuídas de forma fragmentada
    \item Líderes regionais de partidos pequenos têm poder desproporcional
  \end{itemize}

  \item \textbf{Vantagem de Grafos - Travessia Multi-Hop}: Esta análise envolveu:
  \begin{itemize}
    \item Deputado → Partido (1 hop)
    \item Deputado → UF → Região (2 hops)
    \item Agregação e agrupamento
  \end{itemize}

  Em Neo4j: \textbf{2 MATCHs + 2 WITHs}. Em SQL relacional: \textbf{2 JOINs + 2 subqueries + GROUP BY aninhado}. Em MongoDB: \textbf{pipeline de 5+ estágios com \$lookup e \$group}.

  \item \textbf{Próximo Nível de Análise}: Com dados de votações, poderíamos responder:
  \begin{itemize}
    \item "Deputados da mesma região mas partidos diferentes votam juntos em projetos regionais?" (regionalismo > partido?)
    \item "Governadores influenciam votação de bancadas estaduais?" (executivo estadual → legislativo federal)
  \end{itemize}
\end{enumerate}

%----------------------------------------------------------------------------------------
% 7. CONCLUSÃO
%----------------------------------------------------------------------------------------
\section{Conclusão}

\subsection{Desafios do Projeto}

\begin{enumerate}
  \item \textbf{Volume de Dados}: Coletar 1.125 deputados e 1.428 frentes exigiu paginação cuidadosa e controle de rate limiting

  \item \textbf{Qualidade da API}: Alguns endpoints tinham inconsistências (ex: frentes sem aceitar parâmetro \texttt{ordenarPor})

  \item \textbf{Modelagem}: Decidir quais relacionamentos incluir dado o tempo disponível (priorizamos FILIADO\_A e REPRESENTA)

  \item \textbf{Performance de Importação}: 2.247 deputados individuais levaram ~3 minutos para importar (100 por lote)
\end{enumerate}

\subsection{Aprendizados}

\begin{enumerate}
  \item \textbf{Poder de Expressão do Cypher}: Consultas complexas que envolveriam múltiplos JOINs em SQL ficaram muito mais legíveis

  \item \textbf{Visualização Nativa}: Neo4j Browser permite exploração visual imediata, facilitando descoberta de padrões

  \item \textbf{Flexibilidade de Schema}: Adicionar novos relacionamentos (ex: MEMBRO\_DE) não requer ALTER TABLE, apenas novos MERGE

  \item \textbf{Adequação ao Domínio}: Dados legislativos são \textbf{naturalmente um grafo} (deputados, partidos, votações, frentes). Modelar isso em relacional seria forçado

  \item \textbf{Análises Futuras}: Com votações importadas, poderíamos fazer:
  \begin{itemize}
    \item Detecção de comunidades (deputados que votam sempre juntos)
    \item PageRank (deputados mais "centrais" na rede de coautorias)
    \item Predição de votos (baseado em rede de similaridade)
  \end{itemize}
\end{enumerate}

\subsection{Vantagens de Grafos para Este Domínio}

\begin{table}[h]
\centering
\begin{tabular}{|l|p{8cm}|}
\hline
\textbf{Aspecto} & \textbf{Vantagem} \\
\hline
Relacionamentos N:M & Natural (deputado $\leftrightarrow$ frentes, proposições, comissões) \\
Travessia profunda & Amigos de amigos, coautorias indiretas \\
Consultas semânticas & "Deputados do partido X que votaram com Y em Z" \\
Flexibilidade & Adicionar tipos de nós/relacionamentos sem refatoração \\
Visualização & Exploração visual de clusters e comunidades \\
\hline
\end{tabular}
\caption{Benefícios de grafos para dados legislativos}
\end{table}

\subsection{Trabalhos Futuros}

\begin{itemize}
  \item Expandir para incluir votações nominais (análise de coesão partidária)
  \item Implementar algoritmos de detecção de comunidades
  \item Criar dashboard interativo com visualizações D3.js
  \item Análise temporal: evolução de alianças entre legislaturas
  \item Predição: usar rede atual para prever comportamento em votações futuras
\end{itemize}

%----------------------------------------------------------------------------------------
% REFERÊNCIAS
%----------------------------------------------------------------------------------------
\newpage
\section{Referências}

\begin{sloppypar}
\begin{enumerate}
  \item \textbf{Repositório do projeto}: \url{https://github.com/Edu-Spinelli/bdcd-camera}

  \item \textbf{API Dados Abertos - Câmara dos Deputados}: \url{https://dadosabertos.camara.leg.br/swagger/api.html}

  \item \textbf{Neo4j Documentation}: \url{https://neo4j.com/docs/}

  \item \textbf{Cypher Query Language}: \url{https://neo4j.com/docs/cypher-manual/current/}

  \item \textbf{Neo4j Python Driver}: \url{https://neo4j.com/docs/python-manual/current/}

  \item Robinson, I., Webber, J., \& Eifrem, E. (2015). \textit{Graph Databases: New Opportunities for Connected Data}. O'Reilly Media.

  \item Angles, R., \& Gutierrez, C. (2008). Survey of graph database models. \textit{ACM Computing Surveys}, 40(1), 1-39.
\end{enumerate}
\end{sloppypar}

%----------------------------------------------------------------------------------------
% ANEXOS
%----------------------------------------------------------------------------------------
\newpage
\section*{Anexos}

\subsection*{Anexo A: Estrutura dos Arquivos JSON}

\textbf{Exemplo de deputado}:
\begin{lstlisting}[language=Java]
{
  "id": 220593,
  "nome": "Abilio Brunini",
  "siglaPartido": "PL",
  "siglaUf": "MT",
  "urlFoto": "https://www.camara.leg.br/internet/deputado/bandep/220593.jpg",
  "email": "dep.abiliobrunini@camara.leg.br"
}
\end{lstlisting}

\subsection*{Anexo B: Comandos Úteis do Neo4j}

\begin{lstlisting}[language=Java, caption=Limpar banco]
MATCH (n) DETACH DELETE n
\end{lstlisting}

\begin{lstlisting}[language=Java, caption=Ver índices]
SHOW INDEXES
\end{lstlisting}

\begin{lstlisting}[language=Java, caption=Estatísticas gerais]
CALL db.stats.retrieve('GRAPH COUNTS')
\end{lstlisting}

\subsection*{Anexo C: Link do Repositório}

Todos os scripts (coleta, importação, consultas) estão disponíveis em:

\url{https://github.com/Edu-Spinelli/bdcd-camera}

\end{document}

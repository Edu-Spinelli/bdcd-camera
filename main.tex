%%%%%%%%%%%%%%%%%%%%%%%%%%%%%%%%%%%%%%%%%
% Trabalho 2 - Banco de Dados para Ciência de Dados
% Análise de Dados da Câmara dos Deputados com Neo4j
%%%%%%%%%%%%%%%%%%%%%%%%%%%%%%%%%%%%%%%%%

\documentclass[12pt, a4paper]{article}

% --- Codificação e idiomas ---
\usepackage[utf8]{inputenc}
\usepackage[T1]{fontenc}
\usepackage[portuguese]{babel}
\usepackage{lmodern}

% --- Layout e utilidades ---
\usepackage{geometry}
\usepackage{hyperref}
\usepackage{graphicx}
\usepackage{parskip}
\usepackage{fancyhdr}
\usepackage{microtype}
\usepackage{caption}
\usepackage{tikz}
\usetikzlibrary{shapes,arrows,positioning}

% --- Código / listagens ---
\usepackage{xcolor}
\usepackage{listings}
\usepackage{listingsutf8}

% Geometria
\geometry{
  a4paper,
  total={170mm,257mm},
  left=20mm, top=20mm
}

% Hiperlinks
\hypersetup{
  colorlinks=true, linkcolor=blue, urlcolor=cyan, filecolor=magenta,
  breaklinks=true
}
\def\UrlBreaks{\do\/\do-\do\_\do\.\do\?}

% Microtype
\sloppy
\emergencystretch=3em

% Cabeçalho/Rodapé
\pagestyle{fancy}
\fancyhf{}
\fancyhead[L]{Trabalho 2 - BDCD - Câmara dos Deputados}
\fancyfoot[C]{\thepage}
\setlength{\headheight}{15pt}
\renewcommand{\headrulewidth}{0.4pt}
\renewcommand{\footrulewidth}{0.4pt}

% Cores p/ código
\definecolor{codegreen}{rgb}{0,0.6,0}
\definecolor{codegray}{rgb}{0.5,0.5,0.5}
\definecolor{codepurple}{rgb}{0.58,0,0.82}
\definecolor{backcolour}{rgb}{0.95,0.95,0.92}

% Cores para nós do grafo
\definecolor{nodedeputado}{RGB}{100,149,237}
\definecolor{nodepartido}{RGB}{220,20,60}
\definecolor{nodefrente}{RGB}{50,205,50}
\definecolor{nodeuf}{RGB}{255,165,0}

% Listagens
\lstdefinestyle{mystyle}{
  backgroundcolor=\color{backcolour},
  commentstyle=\color{codegreen},
  keywordstyle=\color{magenta},
  numberstyle=\tiny\color{codegray},
  stringstyle=\color{codepurple},
  basicstyle=\footnotesize\ttfamily,
  columns=fullflexible,
  breaklines=true,
  keepspaces=true,
  numbers=left, numbersep=6pt,
  showstringspaces=false, showtabs=false,
  tabsize=2,
  inputencoding=utf8
}

\lstset{
  style=mystyle,
  literate=
   {á}{{\'a}}1 {Á}{{\'A}}1 {é}{{\'e}}1 {É}{{\'E}}1
   {í}{{\'\i}}1 {Í}{{\'I}}1 {ó}{{\'o}}1 {Ó}{{\'O}}1
   {ú}{{\'u}}1 {Ú}{{\'U}}1 {â}{{\^a}}1 {Â}{{\^A}}1
   {ê}{{\^e}}1 {Ê}{{\^E}}1 {ô}{{\^o}}1 {Ô}{{\^O}}1
   {ã}{{\~a}}1 {Ã}{{\~A}}1 {õ}{{\~o}}1 {Õ}{{\~O}}1
   {ç}{{\c c}}1 {Ç}{{\c C}}1 {à}{{\`a}}1 {À}{{\`A}}1
   {ü}{{\"u}}1 {Ü}{{\"U}}1 {º}{{\textordmasculine}}1
   {ª}{{\textordfeminine}}1 {–}{{--}}1 {—}{{---}}1
}

\lstset{captionpos=b}

%----------------------------------------------------------------------------------------
% CAPA
%----------------------------------------------------------------------------------------
\title{
  \vspace*{\fill}
  \centering
  \Huge\bfseries Análise de Dados Legislativos:\\
  Câmara dos Deputados do Brasil com Neo4j\\[0.5cm]
  \large Trabalho 2 - Banco de Dados para Ciência de Dados\\
  \vspace*{\fill}
}
\author{
  \large Eduardo Spinelli -- RA 800220
}
\date{
  \large Centro de Ciências Exatas e de Tecnologia\\
  Departamento de Computação\\
  \textbf{São Carlos - SP}\\
  \today
}

\begin{document}
\maketitle
\thispagestyle{empty}
\newpage

\tableofcontents
\newpage

%----------------------------------------------------------------------------------------
% 1. INTRODUÇÃO E DATASET
%----------------------------------------------------------------------------------------
\section{Introdução e Descrição do Dataset}

\subsection{Contextualização}

Este trabalho apresenta uma análise de dados legislativos da Câmara dos Deputados do Brasil utilizando um banco de dados orientado a grafos (Neo4j). A escolha deste tema permite explorar relações complexas entre deputados, partidos, frentes parlamentares e representação geográfica, demonstrando as vantagens de modelagem em grafo para este tipo de domínio.

\subsection{Fonte dos Dados}

Os dados foram obtidos através da \textbf{API de Dados Abertos da Câmara dos Deputados}, disponível publicamente em:

\url{https://dadosabertos.camara.leg.br/swagger/api.html}

Esta API fornece acesso estruturado a informações oficiais sobre:
\begin{itemize}
  \item Deputados federais (histórico e atuais)
  \item Partidos políticos
  \item Frentes parlamentares (agrupamentos temáticos)
  \item Proposições legislativas (PLs, PECs, etc.)
  \item Votações nominais
  \item Órgãos e comissões
\end{itemize}

\subsection{Período dos Dados}

\textbf{Período analisado}: 2019 a 2024\\
\textbf{Legislaturas}: 56ª (2019-2023) e 57ª (2023-2027)

Este período foi escolhido por conter dados recentes e completos, permitindo análises relevantes do cenário político brasileiro atual.

\subsection{Quantidade de Dados Coletados}

O dataset final contém:

\begin{table}[h]
\centering
\begin{tabular}{|l|r|r|}
\hline
\textbf{Entidade} & \textbf{Quantidade} & \textbf{Tamanho} \\
\hline
Deputados & 1.125 & 848 KB \\
Partidos & 38 & 5.8 KB \\
Frentes Parlamentares & 1.428 & 280 KB \\
Estados (UF) & 27 & - \\
\hline
\textbf{Total de Nós} & \textbf{2.618} & \textbf{1.13 MB} \\
\hline
\end{tabular}
\caption{Quantidade de nós no grafo}
\end{table}

\begin{table}[h]
\centering
\begin{tabular}{|l|r|}
\hline
\textbf{Relacionamento} & \textbf{Quantidade} \\
\hline
FILIADO\_A (Deputado $\rightarrow$ Partido) & 1.125 \\
REPRESENTA (Deputado $\rightarrow$ UF) & 1.125 \\
\hline
\textbf{Total de Relacionamentos} & \textbf{2.250} \\
\hline
\end{tabular}
\caption{Quantidade de relacionamentos no grafo}
\end{table}

\subsection{Detalhes Importantes}

\begin{enumerate}
  \item \textbf{Dados Históricos}: O dataset inclui deputados de múltiplas legislaturas, incluindo aqueles que não estão mais em exercício

  \item \textbf{Mudanças Partidárias}: Alguns deputados podem aparecer com partidos diferentes ao longo do tempo (mudança de filiação)

  \item \textbf{Frentes Temáticas}: As 1.428 frentes parlamentares cobrem temas diversos: agricultura, saúde, educação, infraestrutura, direitos humanos, economia, etc.

  \item \textbf{Representatividade}: Os dados refletem a distribuição real de deputados por estado, respeitando a proporcionalidade populacional brasileira
\end{enumerate}

%----------------------------------------------------------------------------------------
% 2. SCRIPT DE COLETA
%----------------------------------------------------------------------------------------
\section{Script de Coleta de Dados}

\subsection{Tecnologias Utilizadas}

O script de coleta foi desenvolvido em \textbf{Python 3} utilizando:
\begin{itemize}
  \item \texttt{requests}: Para consumir a API REST
  \item \texttt{json}: Para manipulação de dados
  \item \texttt{time}: Para controle de rate limiting
\end{itemize}

\subsection{Arquitetura do Coletor}

O script implementa as seguintes funcionalidades:

\begin{enumerate}
  \item \textbf{Paginação Automática}: Coleta todas as páginas de resultados automaticamente
  \item \textbf{Rate Limiting}: Respeita os limites da API (pausa de 0.5s entre requisições)
  \item \textbf{Retry Mechanism}: Retenta automaticamente em caso de falha
  \item \textbf{Salvamento Incremental}: Salva dados por tipo (partidos, deputados, frentes)
  \item \textbf{Modos de Operação}: Teste, rápido e completo
\end{enumerate}

\subsection{Código Principal}

\begin{lstlisting}[language=Python, caption=Estrutura da classe coletora]
class ColetorDadosCamara:
    def __init__(self, output_dir='dados_camara'):
        self.base_url = "https://dadosabertos.camara.leg.br/api/v2"
        self.headers = {'accept': 'application/json'}
        self.output_dir = output_dir

    def _paginar_requisicao(self, url, params=None):
        """Faz paginacao automatica de requisicoes"""
        todos_dados = []
        pagina = 1
        while True:
            params['pagina'] = pagina
            data = self._fazer_requisicao(url, params)
            if not data or 'dados' not in data:
                break
            todos_dados.extend(data['dados'])
            # Verificar se ha mais paginas
            links = data.get('links', [])
            tem_proxima = any(link.get('rel') == 'next' for link in links)
            if not tem_proxima:
                break
            pagina += 1
        return todos_dados

    def get_deputados(self, data_inicio="2019-01-01", data_fim="2024-12-31"):
        """Coleta todos os deputados do periodo"""
        url = f"{self.base_url}/deputados"
        params = {
            'dataInicio': data_inicio,
            'dataFim': data_fim,
            'ordem': 'ASC',
            'ordenarPor': 'nome'
        }
        return self._paginar_requisicao(url, params)
\end{lstlisting}

\subsection{Execução do Script}

O script pode ser executado em diferentes modos:

\begin{lstlisting}[language=Bash, caption=Execução do coletor]
# Modo teste (poucos dados)
python coletor_dados.py --modo teste

# Modo rapido (dados essenciais)
python coletor_dados.py --modo rapido

# Modo completo (todos os dados)
python coletor_dados.py --modo completo
\end{lstlisting}

\subsection{Saída do Script}

Os dados são salvos em arquivos JSON no diretório \texttt{dados\_camara/}:

\begin{lstlisting}[language=Bash, caption=Estrutura dos arquivos gerados]
dados_camara/
|-- partidos.json          # 38 partidos (5.8 KB)
|-- deputados.json         # 1.125 deputados (848 KB)
|-- frentes.json           # 1.428 frentes (280 KB)
\end{lstlisting}

%----------------------------------------------------------------------------------------
% 3. MODELAGEM DO GRAFO
%----------------------------------------------------------------------------------------
\newpage
\section{Modelagem dos Dados em Grafo}

\subsection{Visão Geral do Modelo}

O modelo de dados representa as relações entre entidades do sistema legislativo brasileiro. Utilizamos 4 tipos de nós e 2 tipos de relacionamentos principais.

\subsection{Diagrama do Modelo com Exemplos}

\begin{figure}[h]
\centering
\begin{tikzpicture}[
  node distance=2.5cm,
  every node/.style={font=\footnotesize},
  deputado/.style={rectangle, rounded corners, draw=nodedeputado, fill=nodedeputado!20, thick, minimum width=3.5cm, minimum height=1.2cm, align=center},
  partido/.style={rectangle, rounded corners, draw=nodepartido, fill=nodepartido!20, thick, minimum width=3cm, minimum height=1.2cm, align=center},
  frente/.style={rectangle, rounded corners, draw=nodefrente, fill=nodefrente!20, thick, minimum width=3.5cm, minimum height=1.2cm, align=center},
  uf/.style={rectangle, rounded corners, draw=nodeuf, fill=nodeuf!20, thick, minimum width=2.5cm, minimum height=1.2cm, align=center},
  rel/.style={->, >=stealth, thick}
]

% Deputados
\node[deputado] (dep1) {\textbf{Deputado}\\id: 220593\\nome: Abilio Brunini};
\node[deputado, below=1.5cm of dep1] (dep2) {\textbf{Deputado}\\id: 204554\\nome: Abílio Santana};

% Partidos
\node[partido, right=3cm of dep1] (part1) {\textbf{Partido}\\sigla: PL\\nome: Partido Liberal};
\node[partido, right=3cm of dep2] (part2) {\textbf{Partido}\\sigla: PSC\\nome: Social Cristão};

% UFs
\node[uf, below right=1cm and 3cm of dep1] (uf1) {\textbf{UF}\\sigla: MT\\regiao: Centro-Oeste};
\node[uf, below=1.5cm of uf1] (uf2) {\textbf{UF}\\sigla: BA\\regiao: Nordeste};

% Frentes
\node[frente, below=4cm of dep2] (frente1) {\textbf{Frente}\\id: 55700\\Integracao Sul-Americana};
\node[frente, right=2cm of frente1] (frente2) {\textbf{Frente}\\id: 54214\\Agropecuaria};

% Relacionamentos
\draw[rel, blue, very thick] (dep1) -- node[above, sloped] {FILIADO\_A} (part1);
\draw[rel, blue, very thick] (dep2) -- node[above, sloped] {FILIADO\_A} (part2);
\draw[rel, orange, very thick] (dep1) -- node[above, sloped] {REPRESENTA} (uf1);
\draw[rel, orange, very thick] (dep2) -- node[above, sloped] {REPRESENTA} (uf2);

\end{tikzpicture}
\caption{Modelo de grafo com 2 exemplos de cada tipo de nó e relacionamento}
\end{figure}

\subsection{Descrição dos Tipos de Nós}

\subsubsection{1. Deputado}
Representa um deputado federal.

\textbf{Propriedades}:
\begin{itemize}
  \item \texttt{id}: Integer (identificador único da API)
  \item \texttt{nome}: String (nome completo)
  \item \texttt{siglaPartido}: String (ex: "PT", "PL")
  \item \texttt{siglaUf}: String (ex: "SP", "BA")
  \item \texttt{urlFoto}: String (URL da foto oficial)
  \item \texttt{email}: String (e-mail institucional)
\end{itemize}

\textbf{Exemplos reais}:
\begin{lstlisting}[language=Java]
(:Deputado {id: 220593, nome: "Abilio Brunini", siglaPartido: "PL", siglaUf: "MT"})
(:Deputado {id: 204554, nome: "Abílio Santana", siglaPartido: "PSC", siglaUf: "BA"})
\end{lstlisting}

\subsubsection{2. Partido}
Representa um partido político brasileiro.

\textbf{Propriedades}:
\begin{itemize}
  \item \texttt{id}: Integer
  \item \texttt{sigla}: String (ex: "PT", "PSDB", "PL")
  \item \texttt{nome}: String (nome completo do partido)
  \item \texttt{uri}: String (endpoint da API)
\end{itemize}

\textbf{Exemplos reais}:
\begin{lstlisting}[language=Java]
(:Partido {id: 36768, sigla: "PT", nome: "Partido dos Trabalhadores"})
(:Partido {id: 37906, sigla: "PL", nome: "Partido Liberal"})
\end{lstlisting}

\subsubsection{3. Frente}
Representa uma frente parlamentar (agrupamento temático de deputados).

\textbf{Propriedades}:
\begin{itemize}
  \item \texttt{id}: Integer
  \item \texttt{titulo}: String (nome da frente)
  \item \texttt{idLegislatura}: Integer (56 ou 57)
  \item \texttt{uri}: String
\end{itemize}

\textbf{Exemplos reais}:
\begin{lstlisting}[language=Java]
(:Frente {id: 55700, titulo: "Frente Parlamentar Mista pela Integracao Sul-Americana", idLegislatura: 57})
(:Frente {id: 54214, titulo: "Frente Parlamentar da Agropecuaria", idLegislatura: 56})
\end{lstlisting}

\subsubsection{4. UF}
Representa um estado brasileiro.

\textbf{Propriedades}:
\begin{itemize}
  \item \texttt{sigla}: String (ex: "SP", "RJ")
  \item \texttt{nome}: String (ex: "São Paulo")
  \item \texttt{regiao}: String (ex: "Sudeste", "Nordeste")
\end{itemize}

\textbf{Exemplos reais}:
\begin{lstlisting}[language=Java]
(:UF {sigla: "SP", nome: "Sao Paulo", regiao: "Sudeste"})
(:UF {sigla: "BA", nome: "Bahia", regiao: "Nordeste"})
\end{lstlisting}

\subsection{Descrição dos Relacionamentos}

\subsubsection{1. FILIADO\_A}
\textbf{De}: Deputado $\rightarrow$ Partido\\
\textbf{Tipo}: Unidirecional, N:1\\
\textbf{Descrição}: Indica a filiação partidária do deputado.

\textbf{Exemplos reais}:
\begin{lstlisting}[language=Java]
(:Deputado {nome: "Abilio Brunini"})-[:FILIADO_A]->(:Partido {sigla: "PL"})
(:Deputado {nome: "Abílio Santana"})-[:FILIADO_A]->(:Partido {sigla: "PSC"})
\end{lstlisting}

\subsubsection{2. REPRESENTA}
\textbf{De}: Deputado $\rightarrow$ UF\\
\textbf{Tipo}: Unidirecional, N:1\\
\textbf{Descrição}: Indica qual estado o deputado representa.

\textbf{Exemplos reais}:
\begin{lstlisting}[language=Java]
(:Deputado {nome: "Abilio Brunini"})-[:REPRESENTA]->(:UF {sigla: "MT"})
(:Deputado {nome: "Abílio Santana"})-[:REPRESENTA]->(:UF {sigla: "BA"})
\end{lstlisting}

\subsection{Justificativa da Modelagem}

A modelagem em grafo é ideal para este domínio por:

\begin{enumerate}
  \item \textbf{Relações Naturais}: Deputados, partidos e estados possuem conexões naturais que grafos expressam melhor que tabelas

  \item \textbf{Consultas de Travessia}: Perguntas como "deputados do mesmo partido em estados diferentes" são triviais em grafo

  \item \textbf{Flexibilidade}: Adicionar novos tipos de relacionamentos (MEMBRO\_DE, VOTOU) não requer ALTER TABLE

  \item \textbf{Performance}: Consultas multi-hop são otimizadas por índices de adjacência
\end{enumerate}

%----------------------------------------------------------------------------------------
% 4. SCRIPT DE IMPORTAÇÃO
%----------------------------------------------------------------------------------------
\section{Script de Importação para Neo4j}

\subsection{Conexão com Neo4j Aura}

Utilizamos o \textbf{driver oficial Python do Neo4j} para importar os dados:

\begin{lstlisting}[language=Python, caption=Conexão com Neo4j Aura]
from neo4j import GraphDatabase

class ImportadorNeo4jAura:
    def __init__(self):
        self.URI = "neo4j+s://2a3ba8da.databases.neo4j.io"
        self.AUTH = ("neo4j", "senha_segura")
        self.driver = GraphDatabase.driver(self.URI, auth=self.AUTH)

    def criar_constraints(self):
        with self.driver.session() as session:
            session.run("""
                CREATE CONSTRAINT deputado_id IF NOT EXISTS
                FOR (d:Deputado) REQUIRE d.id IS UNIQUE
            """)
            session.run("""
                CREATE CONSTRAINT partido_id IF NOT EXISTS
                FOR (p:Partido) REQUIRE p.id IS UNIQUE
            """)
\end{lstlisting}

\subsection{Importação dos Dados}

\begin{lstlisting}[language=Python, caption=Importação de deputados]
def importar_deputados(self):
    deputados = self.carregar_json('dados_camara/deputados.json')

    with self.driver.session() as session:
        for dep in deputados:
            query = """
            MERGE (d:Deputado {id: $id})
            SET d.nome = $nome,
                d.siglaPartido = $siglaPartido,
                d.siglaUf = $siglaUf
            """
            session.run(query,
                id=dep['id'],
                nome=dep['nome'],
                siglaPartido=dep.get('siglaPartido'),
                siglaUf=dep.get('siglaUf'))

        # Criar relacionamentos
        session.run("""
            MATCH (d:Deputado)
            WHERE d.siglaPartido IS NOT NULL
            MATCH (p:Partido {sigla: d.siglaPartido})
            MERGE (d)-[:FILIADO_A]->(p)
        """)
\end{lstlisting}

\subsection{Execução da Importação}

\begin{lstlisting}[language=Bash, caption=Comando de importação]
python importar_aura.py --limpar
\end{lstlisting}

\textbf{Resultado}:
\begin{lstlisting}[language=Bash]
[OK] Conectado ao Neo4j Aura com sucesso!
[AVISO] Limpando banco de dados...
[OK] Banco limpo

========================================
IMPORTACAO PARA NEO4J AURA
========================================
[OK] 27 UFs criadas
[OK] 38 partidos importados
[OK] 1.125 deputados importados
[OK] 1.428 frentes importadas
[OK] Relacionamentos FILIADO_A criados
[OK] Relacionamentos REPRESENTA criados

[SUCESSO] IMPORTACAO CONCLUIDA!
\end{lstlisting}

%----------------------------------------------------------------------------------------
% 5. CONSULTA COM TODOS OS DADOS
%----------------------------------------------------------------------------------------
\section{Consulta com Todos os Dados Importados}

\subsection{Visualização Geral do Grafo}

\begin{lstlisting}[language=Java, caption=Consulta de todos os nós]
MATCH (n)
RETURN n
LIMIT 100
\end{lstlisting}

\textit{[Esta consulta retorna uma visualização gráfica de 100 nós conectados por seus relacionamentos no Neo4j Browser]}

\subsection{Estatísticas do Grafo}

\begin{lstlisting}[language=Java, caption=Contagem por tipo de nó]
MATCH (n)
RETURN labels(n)[0] AS Tipo, count(n) AS Quantidade
ORDER BY Quantidade DESC
\end{lstlisting}

\textbf{Resultado}:

\begin{table}[h]
\centering
\begin{tabular}{|l|r|}
\hline
\textbf{Tipo} & \textbf{Quantidade} \\
\hline
Frente & 1.428 \\
Deputado & 1.125 \\
Partido & 38 \\
UF & 27 \\
\hline
\textbf{TOTAL} & \textbf{2.618 nós} \\
\hline
\end{tabular}
\caption{Estatísticas gerais do grafo}
\end{table}

%----------------------------------------------------------------------------------------
% 6. CINCO CONSULTAS E ANÁLISES
%----------------------------------------------------------------------------------------
\newpage
\section{Cinco Consultas para Insights}

\subsection{Consulta 1: Distribuição de Deputados por Partido}

\subsubsection{Pergunta de Negócio}
\textbf{"Qual a distribuição de deputados entre os partidos? Quais partidos têm maior representatividade no Congresso?"}

\subsubsection{Código Cypher}
\begin{lstlisting}[language=Java, caption=Deputados por partido]
MATCH (p:Partido)<-[:FILIADO_A]-(d:Deputado)
WITH p, count(d) AS numDeputados
RETURN p.sigla AS Partido,
       p.nome AS NomeCompleto,
       numDeputados AS NumeroDeputados
ORDER BY numDeputados DESC
LIMIT 15
\end{lstlisting}

\subsubsection{Resultado}

\begin{table}[h]
\centering
\small
\begin{tabular}{|l|p{6cm}|r|}
\hline
\textbf{Sigla} & \textbf{Nome Completo} & \textbf{Deputados} \\
\hline
PL & Partido Liberal & 128 \\
PP & Progressistas & 118 \\
PT & Partido dos Trabalhadores & 111 \\
UNIÃO & União Brasil & 100 \\
PSD & Partido Social Democrático & 99 \\
MDB & Movimento Democrático Brasileiro & 96 \\
REPUBLICANOS & Republicanos & 80 \\
PSDB & Partido da Social Democracia Brasileira & 58 \\
PSB & Partido Socialista Brasileiro & 49 \\
PDT & Partido Democrático Trabalhista & 41 \\
PODE & Podemos & 36 \\
PSOL & Partido Socialismo e Liberdade & 18 \\
DEM & Democratas & 18 \\
PTB & Partido Trabalhista Brasileiro & 17 \\
SOLIDARIEDADE & Solidariedade & 17 \\
\hline
\end{tabular}
\caption{Top 15 partidos por número de deputados}
\end{table}

\subsubsection{Análise dos Resultados}

Esta consulta revela a \textbf{fragmentação partidária} brasileira e a força numérica de cada legenda no Congresso.

\textbf{Insights principais}:

O domínio do "Centrão" é evidente: partidos de centro e centro-direita (PL, PP, União Brasil, PSD, MDB, Republicanos) juntos possuem mais de 600 deputados, representando aproximadamente 55\% do total. Essa concentração demonstra sua força como bloco de sustentação governamental e explica historicamente porque esses partidos são decisivos em votações importantes.

A polarização PT vs PL também fica clara nos dados. O PL (direita/centro-direita) com 128 deputados e o PT (esquerda) com 111 deputados representam os dois maiores partidos e refletem a divisão ideológica do país nos últimos anos. A diferença de apenas 17 deputados entre eles mostra um equilíbrio de forças entre os polos opostos do espectro político.

A fragmentação extrema é outro ponto crucial: nenhum partido sozinho tem maioria absoluta (necessário 257 deputados de um total de 513). Os 15 maiores partidos mostram uma distribuição relativamente equilibrada, o que torna a governabilidade um desafio permanente. Qualquer governo precisa formar coalizões amplas e negociar com múltiplos partidos para aprovar projetos, especialmente emendas constitucionais que requerem 2/3 dos votos (342 deputados).

\subsection{Consulta 2: Geografia Política - Distribuição Regional}

\subsubsection{Pergunta de Negócio}
\textbf{"Como se distribui geograficamente a representação política? Quais regiões têm mais peso no Congresso e como os estados se comparam?"}

\subsubsection{Código Cypher}
\begin{lstlisting}[language=Java, caption=Deputados por região e top estados]
// Primeiro: Distribuicao por regiao
MATCH (uf:UF)<-[:REPRESENTA]-(d:Deputado)
WITH uf.regiao AS Regiao, count(d) AS numDeputados
RETURN Regiao, numDeputados AS TotalDeputados
ORDER BY numDeputados DESC

// Segundo: Top 10 estados
MATCH (uf:UF)<-[:REPRESENTA]-(d:Deputado)
WITH uf, count(d) AS numDeputados
RETURN uf.sigla AS Estado,
       uf.nome AS NomeEstado,
       uf.regiao AS Regiao,
       numDeputados AS NumDeputados
ORDER BY numDeputados DESC
LIMIT 10
\end{lstlisting}

\subsubsection{Resultado}

\begin{table}[h]
\centering
\begin{tabular}{|l|r|r|}
\hline
\textbf{Região} & \textbf{Deputados} & \textbf{\% do Total} \\
\hline
Sudeste & 377 & 33.5\% \\
Nordeste & 330 & 29.3\% \\
Sul & 159 & 14.1\% \\
Norte & 157 & 14.0\% \\
Centro-Oeste & 102 & 9.1\% \\
\hline
\textbf{TOTAL} & \textbf{1.125} & \textbf{100\%} \\
\hline
\end{tabular}
\caption{Distribuição de deputados por região}
\end{table}

\begin{table}[h]
\centering
\begin{tabular}{|l|p{3.5cm}|l|r|}
\hline
\textbf{UF} & \textbf{Estado} & \textbf{Região} & \textbf{Deputados} \\
\hline
SP & São Paulo & Sudeste & 144 \\
RJ & Rio de Janeiro & Sudeste & 109 \\
MG & Minas Gerais & Sudeste & 102 \\
BA & Bahia & Nordeste & 69 \\
PR & Paraná & Sul & 61 \\
RS & Rio Grande do Sul & Sul & 58 \\
CE & Ceará & Nordeste & 53 \\
PE & Pernambuco & Nordeste & 52 \\
MA & Maranhão & Nordeste & 49 \\
SC & Santa Catarina & Sul & 40 \\
\hline
\end{tabular}
\caption{Top 10 estados com mais deputados}
\end{table}

\subsubsection{Análise dos Resultados}

Esta consulta revela o \textbf{peso político} de cada região e como a população se reflete (ou não) na representação legislativa.

O domínio do Sudeste é incontestável: com 377 deputados (33.5\% do total), a região tem poder de veto sobre qualquer matéria que exija maioria absoluta. São Paulo sozinho, com 144 deputados (12.8\% da Câmara), tem mais representantes que toda a região Centro-Oeste. Essa concentração reflete parcialmente a população, mas com distorções importantes devido aos limites constitucionais (mínimo 8 e máximo 70 deputados por estado).

A sub-representação e super-representação são fenômenos importantes visíveis nos dados. São Paulo tem 22\% da população brasileira mas apenas 12.8\% dos deputados, sendo sub-representado devido ao teto constitucional. Por outro lado, estados pequenos do Norte e Nordeste têm proporcionalmente mais deputados devido ao mínimo constitucional de 8 deputados, criando uma super-representação relativa.

O equilíbrio Sudeste-Nordeste é estratégico: juntas, essas duas regiões controlam 62.8\% da Câmara (707 deputados). Qualquer projeto de lei nacional precisa de apoio significativo de pelo menos uma dessas regiões para ser aprovado. Isso explica porque políticas regionais para Nordeste e Sudeste sempre recebem atenção especial dos governos.

A bancada paulista merece destaque especial: com 144 deputados, se votasse de forma coesa, poderia sozinha barrar emendas constitucionais (que necessitam 342 votos = 2/3 da Casa). No entanto, a fragmentação partidária dentro de SP reduz esse poder potencial, mostrando como partido supera geografia na maioria das votações.

\subsection{Consulta 3: Força Partidária por Região}

\subsubsection{Pergunta de Negócio}
\textbf{"Partidos têm força regional ou presença nacional? Existem 'feudos' partidários em determinadas regiões?"}

\subsubsection{Código Cypher}
\begin{lstlisting}[language=Java, caption=Distribuição partidária regional]
MATCH (d:Deputado)-[:FILIADO_A]->(p:Partido)
MATCH (d)-[:REPRESENTA]->(uf:UF)
WITH uf.regiao AS Regiao, p.sigla AS Partido, count(d) AS numDeputados
WHERE numDeputados >= 5
RETURN Regiao, Partido, numDeputados AS NumDeputados
ORDER BY Regiao, numDeputados DESC
\end{lstlisting}

\subsubsection{Resultado (Resumido por Região)}

\begin{table}[h]
\centering
\small
\begin{tabular}{|l|l|r|}
\hline
\textbf{Região} & \textbf{Partido} & \textbf{Deputados} \\
\hline
\multirow{6}{*}{Centro-Oeste} & PL & 16 \\
& MDB & 14 \\
& PP & 10 \\
& REPUBLICANOS & 9 \\
& PSDB & 9 \\
& PT & 8 \\
\hline
\multirow{6}{*}{Nordeste} & PP & 39 \\
& UNIÃO & 35 \\
& PL & 33 \\
& PT & 31 \\
& PSD & 28 \\
& PSB & 24 \\
\hline
\multirow{5}{*}{Sul} & PL & 27 \\
& PT & 20 \\
& PP & 18 \\
& MDB & 14 \\
& PSDB & 13 \\
\hline
\multirow{6}{*}{Sudeste} & PL & 48 \\
& PT & 40 \\
& UNIÃO & 38 \\
& PP & 34 \\
& PSD & 32 \\
& REPUBLICANOS & 29 \\
\hline
\end{tabular}
\caption{Força partidária por região (partidos com 5+ deputados)}
\end{table}

\subsubsection{Análise dos Resultados}

Esta consulta demonstra \textbf{padrões de distribuição geográfica} dos partidos, revelando estratégias nacionais versus regionais.

O PL demonstra força nacional inquestionável: está entre os 3 principais partidos em TODAS as regiões (48 no Sudeste, 33 no Nordeste, 27 no Sul, 16 no Centro-Oeste). Essa distribuição homogênea indica capilaridade nacional bem estabelecida, resultado de uma estratégia de crescimento que incorporou outros partidos menores de direita nos últimos anos. Essa presença nacional dá ao PL vantagem em negociações governamentais, pois não depende de uma única região.

O PP revela um "feudo" nordestino: dos 118 deputados totais do PP, 39 (33\%) são do Nordeste - uma concentração desproporcional. Isso indica forte enraizamento regional, provavelmente ligado a lideranças locais históricas e máquinas políticas estaduais consolidadas. Partidos com bases regionais fortes como essa são vulneráveis a mudanças de liderança local, mas têm poder de negociação aumentado em projetos regionais.

O PSB é claramente um partido regional: dos 49 deputados, 24 (49\%) são do Nordeste. Pernambuco, berço histórico do partido, concentra boa parte dessa bancada. Essa regionalização limita o poder de barganha nacional do PSB, mas o torna essencial em votações sobre políticas nordestinas.

A polarização geográfica PT vs PL é interessante: ambos têm presença nacional, mas com nuances. O PL é mais forte no Sudeste (48) e Sul (27), enquanto o PT, curiosamente, tem apenas 31 no Nordeste - região historicamente associada ao partido. Isso pode indicar fragmentação da esquerda em legendas menores ou mudanças no perfil eleitoral da região.

As implicações estratégicas são claras: partidos nacionais (PL, UNIÃO, PP) podem ceder em uma região e compensar em outra durante negociações, enquanto partidos regionais (PSB, alguns pequenos) têm menos flexibilidade mas podem ser decisivos em temas geográficos específicos.

\subsection{Consulta 4: Análise de Frentes Temáticas}

\subsubsection{Pergunta de Negócio}
\textbf{"Quais são os temas que mobilizam a criação de frentes parlamentares? Quais assuntos têm mais tração legislativa?"}

\subsubsection{Código Cypher}
\begin{lstlisting}[language=Java, caption=Análise de frentes temáticas]
// Buscar frentes de "Defesa" e "Apoio" (temas comuns)
MATCH (f:Frente)
WHERE f.titulo CONTAINS 'Defesa' OR f.titulo CONTAINS 'Apoio'
RETURN f.titulo AS Frente, f.idLegislatura AS Legislatura
ORDER BY f.titulo
LIMIT 30
\end{lstlisting}

\subsubsection{Resultado (Amostra)}

\begin{table}[h]
\centering
\footnotesize
\begin{tabular}{|p{10cm}|c|}
\hline
\textbf{Frente Parlamentar} & \textbf{Leg.} \\
\hline
Frente Parlamentar Conservadora em Defesa da Liberdade & 57 \\
Frente Parlamentar Mista Cristã e em Defesa da Religião & 57 \\
Frente Parlamentar Mista contra o Aborto e em Defesa da Vida & 57 \\
Frente Parlamentar Mista contra o Aborto e em Defesa da Vida & 56 \\
Frente Parlamentar Mista da Defesa Nacional & 55 \\
Frente Parlamentar Mista da Família e Apoio à Vida & 55 \\
Frente Parlamentar Mista de Apoio ao Escotismo no Brasil & 56 \\
Frente Parlamentar Mista de Apoio ao Jovem Aprendiz & 56 \\
Frente Parlamentar Mista de Apoio ao Mercado Imobiliário & 56 \\
Frente Parlamentar da Agropecuária & 56 \\
\hline
\multicolumn{2}{|c|}{\textit{... e mais 1.418 frentes no total}} \\
\hline
\end{tabular}
\caption{Amostra de frentes parlamentares (10 de 1.428)}
\end{table}

\subsubsection{Análise dos Resultados}

Das \textbf{1.428 frentes parlamentares}, observamos padrões temáticos reveladores sobre interesses legislativos.

A quantidade massiva de frentes indica forte organização de lobbies setoriais. Cada frente é um canal formal de influência para interesses específicos - desde setores econômicos (agropecuária, indústria, comércio) até causas sociais (direitos humanos, saúde, educação). O fato de existirem quase 1.500 frentes mostra que praticamente todo segmento organizado da sociedade consegue criar seu grupo de articulação no Congresso.

O agronegócio demonstra organização exemplar: frentes relacionadas a agricultura, pecuária, cooperativismo rural aparecem frequentemente e de forma específica (há frentes para culturas individuais como carnaúba, bambu, cacau). Isso reflete o peso econômico do agronegócio no PIB brasileiro (23-25\%) e explica a influência da "bancada ruralista" - que não é apenas um grupo informal, mas uma rede institucionalizada através dessas frentes.

A fragmentação temática extrema é notável: existem frentes muito específicas que mostram nichos conseguindo organização legislativa. Isso pode indicar: (1) força de lobbies regionais/setoriais pequenos mas bem articulados; (2) estratégia de deputados para mostrar "atuação temática" aos eleitores; (3) mecanismo para direcionamento estratégico de emendas parlamentares para setores específicos.

Temas transversais aparecem repetidamente em diferentes legislaturas: frentes sobre "Defesa da Família", "Defesa da Vida", "Defesa Nacional" são recriadas a cada legislatura, indicando agendas permanentes e não conjunturais. Essas frentes "perenes" tendem a ter maior coesão e influência real em votações do que frentes oportunísticas.

A sobreposição de frentes similares é intrigante: várias frentes tratam do mesmo tema (5+ frentes sobre saúde mental, múltiplas sobre agricultura familiar). Isso pode significar: (1) competição entre lideranças pelo mesmo nicho; (2) renovação natural entre legislaturas; (3) fragmentação ideológica dentro do mesmo tema (diferentes abordagens para o mesmo problema).

O insight mais poderoso viria de dados adicionais: se tivéssemos os membros de cada frente (relacionamento MEMBRO\_DE), poderíamos responder perguntas cruciais como "Deputados de partidos opostos em mesmas frentes votam junto em projetos relacionados?" (coesão temática > coesão partidária?) e "Quais deputados são 'hubs' participando de muitas frentes?" (identificando brokers de influência). Esses dados revelariam se frentes realmente influenciam votações ou são apenas sinalização política.

\subsection{Consulta 5: Composição Multi-Dimensional Partido-Região}

\subsubsection{Pergunta de Negócio}
\textbf{"Qual a composição partidária detalhada de cada região? Existem padrões de dominância ou extrema fragmentação?"}

\subsubsection{Código Cypher}
\begin{lstlisting}[language=Java, caption=Análise multidimensional partido-região]
MATCH (d:Deputado)-[:FILIADO_A]->(p:Partido)
MATCH (d)-[:REPRESENTA]->(uf:UF)
WITH uf.regiao AS Regiao,
     p.sigla AS Partido,
     count(d) AS Deputados
WHERE Deputados >= 3
WITH Regiao,
     collect({partido: Partido, qtd: Deputados}) AS distribuicao,
     sum(Deputados) AS totalRegiao
RETURN Regiao,
       totalRegiao AS TotalDeputados,
       size(distribuicao) AS NumPartidos
ORDER BY totalRegiao DESC
\end{lstlisting}

\subsubsection{Resultado}

\begin{table}[h]
\centering
\begin{tabular}{|l|r|r|}
\hline
\textbf{Região} & \textbf{Total Deputados} & \textbf{Partidos (3+)} \\
\hline
Sudeste & 377 & 18 \\
Nordeste & 330 & 17 \\
Sul & 159 & 15 \\
Norte & 157 & 13 \\
Centro-Oeste & 102 & 12 \\
\hline
\end{tabular}
\caption{Fragmentação partidária por região}
\end{table}

\textbf{Distribuição detalhada (Top 3 por região)}:

\begin{table}[h]
\centering
\small
\begin{tabular}{|l|l|r|r|}
\hline
\textbf{Região} & \textbf{Partido} & \textbf{Deputados} & \textbf{\% Regional} \\
\hline
\multirow{3}{*}{Sudeste} & PL & 48 & 12.7\% \\
& PT & 40 & 10.6\% \\
& UNIÃO & 38 & 10.1\% \\
\hline
\multirow{3}{*}{Nordeste} & PP & 39 & 11.8\% \\
& UNIÃO & 35 & 10.6\% \\
& PL & 33 & 10.0\% \\
\hline
\multirow{3}{*}{Sul} & PL & 27 & 17.0\% \\
& PT & 20 & 12.6\% \\
& PP & 18 & 11.3\% \\
\hline
\multirow{3}{*}{Norte} & PL & 20 & 12.7\% \\
& UNIÃO & 17 & 10.8\% \\
& PP & 15 & 9.6\% \\
\hline
\multirow{3}{*}{Centro-Oeste} & PL & 16 & 15.7\% \\
& MDB & 14 & 13.7\% \\
& PP & 10 & 9.8\% \\
\hline
\end{tabular}
\caption{Top 3 partidos por região}
\end{table}

\subsubsection{Análise dos Resultados}

Esta consulta combina \textbf{três dimensões} (Deputado, Partido, Região) revelando o "mapa político" do Brasil.

Não há hegemonia regional absoluta: nenhuma região é dominada por um único partido. Mesmo o PL, partido líder nacional, tem no máximo 17\% em uma região (Sul). Mesmo regiões com histórico de liderança forte (como PT no Sul durante governos Lula) hoje mostram distribuição fragmentada entre 12-18 partidos significativos por região. Essa pulverização impede qualquer partido de ter controle regional absoluto.

A fragmentação como constante nacional é o padrão dominante: mesmo a região com maior concentração partidária ainda tem 12 partidos com pelo menos 3 deputados. O Sudeste, região mais populosa, tem 18 partidos relevantes. Isso explica estruturalmente a necessidade do "presidencialismo de coalizão" brasileiro - não é uma escolha, mas uma consequência inevitável da fragmentação partidária nacional replicada em todas as regiões.

O poder desproporcional de partidos médios emerge claramente: em votações apertadas, partidos com 10-20 deputados podem ser decisivos. Por exemplo, no Nordeste, a diferença entre o 1º (PP, 39) e o 6º colocado (PSB, 24) é de apenas 15 deputados - uma margem pequena que torna o 6º colocado potencialmente decisivo em votações regionais. Isso dá poder de barganha desproporcional a partidos médios bem posicionados.

A consistência do PL como força nacional é notável: aparece no top 3 de TODAS as 5 regiões, com percentuais relativamente homogêneos (10-17\%). Essa distribuição equilibrada dá ao partido flexibilidade estratégica - pode ceder em uma região e compensar em outra. Partidos regionalizados (como PSB no Nordeste) não têm essa flexibilidade.

As implicações para governabilidade são profundas: (1) Governo Federal precisa negociar com múltiplos partidos em CADA região para aprovar projetos regionais; (2) Emendas parlamentares são distribuídas de forma extremamente fragmentada, reduzindo capacidade de investimento coordenado; (3) Líderes regionais de partidos médios têm poder desproporcional, podendo bloquear políticas específicas.

O insight mais valioso dessa análise multi-dimensional: a fragmentação não é apenas nacional, mas se replica em cada região. Isso significa que não há "redutos" onde algum governo possa ter maioria confortável - a necessidade de coalizões amplas é uma constante geográfica, não apenas partidária. Qualquer reforma política que busque reduzir fragmentação precisa atuar simultaneamente em todas as regiões.

%----------------------------------------------------------------------------------------
% 7. CONCLUSÃO
%----------------------------------------------------------------------------------------
\newpage
\section{Conclusão}

\subsection{Desafios do Projeto}

\begin{enumerate}
  \item \textbf{Volume de Dados}: Coletar 1.125 deputados e 1.428 frentes exigiu paginação cuidadosa e controle de rate limiting para não sobrecarregar a API

  \item \textbf{Qualidade da API}: Alguns endpoints tinham inconsistências (ex: frentes sem aceitar parâmetro \texttt{ordenarPor}, causando erro 400)

  \item \textbf{Decisões de Modelagem}: Escolher quais relacionamentos incluir dado o tempo disponível foi crítico - priorizamos FILIADO\_A e REPRESENTA para análises fundamentais

  \item \textbf{Performance de Importação}: Importar 2.618 nós e 2.250 relacionamentos levou cerca de 3 minutos com processamento em lote

  \item \textbf{Limpeza de Dados}: Alguns deputados tinham dados incompletos (sem partido ou UF), exigindo tratamento de valores nulos
\end{enumerate}

\subsection{Aprendizados}

\begin{enumerate}
  \item \textbf{Poder de Expressão do Cypher}: Consultas complexas que envolveriam múltiplos JOINs em SQL ficaram muito mais legíveis e naturais em Cypher. A sintaxe pattern-matching é intuitiva.

  \item \textbf{Visualização Nativa}: O Neo4j Browser permite exploração visual imediata do grafo, facilitando descoberta de padrões que não seriam óbvios em tabelas

  \item \textbf{Flexibilidade de Schema}: Adicionar novos tipos de nós ou relacionamentos (ex: MEMBRO\_DE para conectar deputados a frentes) não requer ALTER TABLE complexos, apenas novos MERGEs

  \item \textbf{Adequação ao Domínio}: Dados legislativos são \textbf{naturalmente um grafo} - deputados conectados a partidos, estados, frentes, proposições. Forçar isso em modelo relacional seria artificial

  \item \textbf{Performance de Travessia}: Consultas que cruzam múltiplas entidades (deputado → partido → região) são otimizadas por índices de adjacência nativos de grafos

  \item \textbf{Análise de Insights}: A capacidade de fazer perguntas complexas ("partidos com forte presença regional") e obter respostas rapidamente acelera análise exploratória
\end{enumerate}

\subsection{Vantagens de Grafos para Este Domínio}

\begin{table}[h]
\centering
\begin{tabular}{|l|p{9cm}|}
\hline
\textbf{Aspecto} & \textbf{Vantagem do Grafo} \\
\hline
Relacionamentos N:M & Natural (deputado $\leftrightarrow$ frentes, proposições, comissões) sem tabelas intermediárias \\
\hline
Travessia profunda & "Amigos de amigos", coautorias indiretas, influência transitiva \\
\hline
Consultas semânticas & "Deputados do partido X que votaram com Y em projetos sobre Z" \\
\hline
Flexibilidade & Adicionar novos tipos de nós/relacionamentos sem refatoração de schema \\
\hline
Visualização & Exploração visual de clusters, comunidades e hubs de influência \\
\hline
Performance & Travessias multi-hop são $O(1)$ por salto vs $O(n \log n)$ de JOINs \\
\hline
\end{tabular}
\caption{Benefícios de grafos para dados legislativos}
\end{table}

\subsection{Trabalhos Futuros}

Com mais tempo e dados, o projeto poderia ser expandido para:

\begin{itemize}
  \item \textbf{Votações Nominais}: Importar votos individuais para analisar coesão partidária vs regionalismo

  \item \textbf{Membros de Frentes}: Adicionar relacionamento MEMBRO\_DE para identificar deputados influentes (hubs) e medir coesão temática

  \item \textbf{Detecção de Comunidades}: Aplicar algoritmos de Louvain para identificar clusters de deputados que votam juntos independentemente de partido

  \item \textbf{PageRank}: Calcular centralidade de deputados em redes de coautorias de projetos

  \item \textbf{Predição de Votos}: Usar características do grafo (conexões, comunidades) para prever comportamento em votações futuras

  \item \textbf{Análise Temporal}: Estudar evolução de alianças e mudanças partidárias entre legislaturas

  \item \textbf{Dashboard Interativo}: Criar interface web com D3.js para visualização dinâmica do grafo
\end{itemize}

%----------------------------------------------------------------------------------------
% REFERÊNCIAS
%----------------------------------------------------------------------------------------
\newpage
\section{Referências}

\begin{sloppypar}
\begin{enumerate}
  \item \textbf{Repositório do projeto}: \url{https://github.com/Edu-Spinelli/bdcd-camera}

  \item \textbf{API Dados Abertos - Câmara dos Deputados}: \url{https://dadosabertos.camara.leg.br/swagger/api.html}

  \item \textbf{Neo4j Documentation}: \url{https://neo4j.com/docs/}

  \item \textbf{Cypher Query Language}: \url{https://neo4j.com/docs/cypher-manual/current/}

  \item \textbf{Neo4j Python Driver}: \url{https://neo4j.com/docs/python-manual/current/}

  \item Robinson, I., Webber, J., \& Eifrem, E. (2015). \textit{Graph Databases: New Opportunities for Connected Data}. O'Reilly Media.

  \item Angles, R., \& Gutierrez, C. (2008). Survey of graph database models. \textit{ACM Computing Surveys}, 40(1), 1-39.
\end{enumerate}
\end{sloppypar}

\end{document}
